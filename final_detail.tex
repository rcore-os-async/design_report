\documentclass[UTF-8]{ctexart}
\usepackage{listing}
\usepackage[cache=false]{minted}
\usepackage{tikz}

\title{First-class async in rCore}
\author{g01 - 王宇逸, 刘晓义}
\date{2020-6}

\begin{document}
\maketitle

\section{设计背景}
Rust 标准库中对异步操作的建模是\textbf{无栈协程},对应语言标准中的 \texttt{struct core::future::Future}。此外,Rust 对基于 Future 的异步操作提供了语言特性层面的支持,通过同步形式的代码编写异步程序,如代码片段 \ref{src:async} 所示。

\begin{listing}[H]
    \begin{minted}{rust}
    async fn foo() -> u64 {
      let result = bar().await;
      println!("Got result: {}", result);
      result
    }
  \end{minted}
    \caption{Async/await example}
    \label{src:async}
\end{listing}

类似于使用标准库中的线程建模并发操作需要操作系统支持,执行无栈协程需要使用 Executor 驱动协程。一般而言,Rust 的协程是针对用户态 IO 密集程序设计的,因此社区所提供的 Executor 绝大多数也都是为这一目的设计。我们在 \ref{sec:executor} 一节中具体讨论 Executor 的问题。

虽然如此,我们认为能够用 Async 和 Future 来抽象内核中的并行工作,因为这种抽象方式有以下的优点:

\paragraph{静态内存分析:} 如果使用 async 块生成 Future 的话,会由编译器静态决定它所占用的空间,这取决于 yield point 集的内存占用上界。由于协程并不是任何地方都可以被打断,因此可以节约空间。

\paragraph{内建消息通知机制:} 一般而言,在内核中实现 Executor 需要实现一个通知机制,将阻塞的线程唤醒。Future 定义了标准的消息通知机制,这可以方便内核各个模块解耦。

\paragraph{撤销机制:} Drop 意味着撤销,因此撤销并行或者组合的异步操作非常简单。强制各个异步服务的提供方支持撤销,方便优化。

\vspace{1em}
因此,我们认为新的抽象能为操作系统内核提供更好的性能和代码可读性,包括:
\begin{itemize}
    \item 处理同步互斥
    \item 驱动用户线程
    \item 处理中断
    \item 实现多核 CPU 支持
\end{itemize}

\section{设计完成示意}
\tikzstyle{block} = [rectangle, draw, fill=blue!20, text centered, rounded corners, minimum width=2.5cm]
\tikzstyle{line} = [draw, -latex]

\begin{tikzpicture}
    \node[block](wheel){Timing Wheel};
    \node[block, right of=wheel, node distance=3cm](executor){Executor};
    \node[block, above of=executor](queueue){MPMC Queue};
    \node[block, right of=executor, node distance=3cm](timeout){Futures};
    \node[block, right of=timeout, node distance=3cm](tutorial){rcore-tutorial};
    \node[block, below of=executor](zcore){zCore};
    \node[block, right of=zcore, node distance=3cm](smp){SMP};

    \path[line](queueue)--(executor);
    \path[line](wheel)--(executor);
    \path[line](executor)--(timeout);
    \path[line](timeout)--(tutorial);
    \path[line, dashed](executor)--(zcore);
    \path[line](zcore)--(smp);
\end{tikzpicture}

\section{zCore}
\subsection{Why zCore}
在经过考察后,我们认为在 zCore 基础上继续 async os 的开发工作是可行的。

\paragraph{目标相同:}zCore 在 async rust 的基础上构建系统内核。
\paragraph{思路相同:}使用 future 的概念替代用户进程/线程,详见下面的分析。
\paragraph{上手较易:}处于初级活跃开发阶段,而且还没有 SMP 支持。

\subsection{async in zCore}

\begin{listing}[H]
    \begin{minted}{rust}
fn spawn(thread: Arc<Thread>) {
  // ...
  let future = async move {
    kernel_hal::Thread::set_tid(thread.id(),
      thread.proc().id());
    loop {
      // ...
      match cx.trap_num {
        0x100 => exit = handle_syscall(&thread,
          &mut cx.general).await,
        0x20..=0x3f => {
          // Interrupt/Timer
        }
        0xe => {
          // Page fault
        }
        _ => panic!(
          "not supported interrupt from user mode. {:#x?}",
          cx),
      }
      // ...
    }
    // ...
  };
  kernel_hal::Thread::spawn(Box::pin(future), vmtoken);
}
    \end{minted}
    \caption{Root Future in zCore}
    \label{src:zcore}
\end{listing}

可以看出,zCore 的用户线程实际上是一个根级(root)Future,这个 Future 直接被提交到 Executor 中进行调度。

用户线程 Future 的工作是一个无限循环,处理各种由用户态进入内核态的可能发生的事件,包括系统调用、中断和页面错误。

\section{Executor}
\label{sec:executor}
\subsection{作用与保证}
Executor 担任了内核中的 Scheduler 的功能。在重用各种 Scheduler 的算法的同时,为 async 机制提供 runtime。

使用 Executor 的优点包括:
\paragraph{Waker 不遗失:}Waker 本质上是一个更加复杂的异步回调(callback)。
在传统的中断处理机制中,为了确保实现正确,可能会暂时关闭中断,这就可能导致对于中断响应的遗失。
使用 async 机制,尽管不能减少处理中断的耗时,但是所有的任务都会被添加到队列。
\paragraph{一致性:}Executor 对于任何 async 任务的调度是一致的,不论是时钟中断、外设中断、页面错误(page fault)、系统调用,
在 Executor 看来都是一个个任务。这给更高层次的抽象,如 async-task 包的工作,提供了便利。
\paragraph{解耦合:}Executor 作为底层 runtime,和上层的使用是解耦合的。
一个使用 async 机制的操作系统可以根据需要使用不同的 Executor,而 Executor 的实现也不需要考虑操作系统的实现细节。
考虑到未来商业操作系统的发展方向之一,也是将与 CPU 等硬件相关的调度工作与操作系统的上层实现解耦,让硬件厂商提供相应的调度实现,来得到更高的性能。
Executor 很好地实现了这一目标。
\subsection{Stealing Queue}
我们实现的 Executor 的基础是一个 MPMC Queue,这是由 Folly 的免锁动态队列改写而来。这样的队列具有如下特点:
\paragraph{动态增长:}在之前的实现中,这是一个静态队列。一旦出现高密度的任务提交,会快速填满队列,导致卡死。
这次的队列继续参考 Folly 的实现,是可以动态并无限增长的。
\paragraph{高并发:}尽管队列增长的过程中,需要一定的同步机制保证正确性。
但是在增长前和增长后,向队列中添加任务是并发免锁的,亦即支持高并发的任务提交。

\section{SMP 支持}
由于 zCore 的 RISC-V 实现并不完全,我们最终决定在 x86\_64 架构下添加 SMP 支持。
在这个过程中我们遇到了许多问题,有些仍待解决。

\paragraph{UEFI 的局限性:}UEFI 支持一种“多任务处理”机制,即提供启动多个 CPU 核心的接口。然而在 UEFI 程序退出时,各个由 BSP 启动的 AP 会被 halt。
也就是说,使用 UEFI 提供好的接口不能直接实现 SMP。
\paragraph{页表的同步:}如果自己实现 bootloader,我们会清楚地掌握实模式、保护模式和长模式下的页表结构和地址,从而可以较为简单地实现页表同步。
然而现在操作系统发展的方向是 UEFI 启动,而从实模式到长模式的工作是 UEFI 完成的。
那么启动的 AP 还要从实模式开始演化,就需要与 UEFI 进行某种配合,或者沿着 UEFI 初始化的路线重新走一遍。
这部分的代码需要手动完成,而且是汇编代码。

\end{document}
