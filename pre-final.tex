\documentclass[UTF-8]{ctexbeamer}
\usetheme{Boadilla}

\usepackage{listing}
\usepackage[cache=false]{minted}

\title{First-class async in rCore}
\subtitle{Pre-Final report}

\author{g01 - 王宇逸, 刘晓义}
\date{2020-3}

\begin{document}
\begin{frame}
  \titlepage
\end{frame}

\begin{frame}
  \frametitle{What is this}

  我们希望能够用 Async 和 Future 来完成内核中的并行工作,包括:
  \begin{itemize}
    \item 处理同步互斥
    \item 驱动用户线程
    \item 处理中断
  \end{itemize}
\end{frame}

\begin{frame}
  \frametitle{What have been done}

  到目前为止的工作包括
  \begin{itemize}
    \item 实现了一些基础的数据结构
    \item 在 rCore tutorial 之上有一个例子
    \item zCore
  \end{itemize}
\end{frame}

\begin{frame}
  \frametitle{What's being done}

   正在为 zCore 添加 SMP 支持

   \pause
   \vspace{1em}

   \textbf{Why?}

   目前 zCore 的 executor (\texttt{Mutex<VecDeque>}) 在单核下已经没有很大的优化空间了。而我们的数据结构是针对多核 Stealing queue 使用环境进行优化。
\end{frame}

\end{document}
