\documentclass[UTF-8]{ctexbeamer}
\usetheme{Boadilla}

\usepackage{listing}
\usepackage{minted}

\title{First-class async in rCore}
\subtitle{设计方案}

\author{g01 - 王宇逸, 刘晓义}
\date{2020-3}

\begin{document}
\begin{frame}
  \titlepage
\end{frame}

\begin{frame}[fragile]
  \frametitle{What are we doing?}

  Rust 有一个叫做 Async 的东西

  \begin{minted}{rust}
    async fn foo() -> u64 {
      let result = bar().await;
      println!("Got result: {}", result);
      result
    }
  \end{minted}

  \vspace{1em}

  我们希望能够用 Async 和 Future 来完成内核中的并行工作,包括:
  \begin{itemize}
    \item 处理同步互斥
    \item 驱动用户线程
    \item 处理中断
  \end{itemize}
\end{frame}

\begin{frame}[fragile]
  \frametitle{But why?}

  \textbf{因为 Future 很便宜。}
  
  使用 \texttt{async fn} 得到的 Future 会由编译器静态计算出所需的空间,然后被放到堆上,这样就不需要对其他线程分配内核栈,也不需要担心内核栈爆炸了。

  \vspace{1em}

  \textbf{因为 Future 很方便。}

  \textbf{内建}了通知机制,同时有很多 Combinator 可以用。如果需要在内核内实现一个请求超时:

  {
    \scriptsize
  \begin{minted}{rust}
    async fn read(f: File, buf: &mut [u8], timeout: Duration)
      -> Result<usize, Error> {
        select! {
          len = f.read_into(buf) => Ok(len),
          () => timer::timeout(timeout) => Err(Error::Timeout),
        }
      }
  \end{minted}
  }
\end{frame}

\begin{frame}
  \frametitle{Previously...}

  有哪些有关的社区工作呢?
  \pause
  \begin{itemize}
    \item \textbf{futures:} 提供了很多 Combinators,虽然和 Scheduler 无关,但是在构建 Primitive Future 的时候,因为没办法直接用 async fn,这些组合子基本是一定要用的。
    \pause
    \item \textbf{Tokio \& async-std \& futures 的 executor:} 依赖 \texttt{std::sync::Mutex} 进行同步,还有 \texttt{std::thread::Thread} 创建进程池。其中后者在 Scheduler 内部连替代都写不出来。
    \pause
    \item \textbf{很多 no-std executor}: 只是在调用线程上同步执行,而且由于没有 Sync primitives,所以也没有异步消息。
    \pause
    \item \textbf{async-task:} 不是一个完整的 Executor,而是 Task 的抽象,处理了 Wake up 相关的逻辑。具体谁来 Poll,以及按照什么顺序 Poll,需要另外完成。
  \end{itemize}

\end{frame}
\begin{frame}
  \frametitle{Previously... Cont.}
  可以用 futures 的组合字和 async-task 的 Task 抽象,部分 Task 的组件 (Waker, etc.) 性能可能稍逊色于手动实现,除此以外还需要实现:
  \begin{itemize}
    \item 最基本的异步单元:Timeout, Mutex, Channel, ...
    \item Run queue 本身
  \end{itemize}
\end{frame}

\begin{frame}[fragile]
  \frametitle{基础设施}

  \begin{itemize}
    \item softirq: 中断队列
    \item timer: 计时器
    \item userspace manager: 将用户态对内核态的接口包装为 Future
    \item synchronization primitives: Mutex, Condvar, Channels...
    \item executor: 执行 Future
  \end{itemize}

\end{frame}

\begin{frame}[fragile]
  \frametitle{基础设施 Cont.}

  这些模块提供以下的 Primitives
  \vspace{1em}

  {
  \scriptsize
  \begin{minted}{rust}
    fn irq::wait_for(int: Interrupt) -> impl Future<Item = InterruptInfo>;
    fn timer::timeout(dur: Duration) -> impl Future<Item = InterruptInfo>;

    enum ut::Event { Syscall(Syscall), TimeUp, Signal(Signal) }
    impl ut::UserThread {
      fn run() -> impl Future<Item = Event>;
    }

    impl<T: Sync> sync::Mutex<T> {
      fn wait_until<F: Fn(&mut self, &mut T) -> bool>(pred: F)
        -> impl Future<Item = Guard<T>>;
    }

    impl exec::Executor {
      fn spawn(fut: Future<Item = ()>);
    }
    fn exec::yield() -> impl Future<Item = ()>;
  \end{minted}
  }
\end{frame}

\begin{frame}[fragile]
  \frametitle{Example}

  \scriptsize
  \begin{minted}{rust}
    async fn create_user_thread(us: UserThread, sig: Signals) {
      match us.run().await {
        Event::TimeUp => yield().await;
        Event::Syscall(Syscall::Exit(code)) => {
          us.notify_waiter(/* ... */);
          return;
        },
        Event::Syscall(Syscall::Sleep(spec)) => {
          let dur: Duration = spec.into();
          timer::timeout(dur).await;
        }
        Event::Signal(Signal::Kill) => {
          us.notify_waiter(/* ... */);
          return ();
        }
        // ...
      }
    }
  \end{minted}
\end{frame}

\begin{frame}
  \frametitle{Progress?}

  \begin{itemize}
    \item 实现了 Timer, Channel, Executor 的底层数据结构
    \item API 设计
    \item 调查了之前社区进行的工作
  \end{itemize}
\end{frame}
  
\end{document}
