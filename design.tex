\documentclass[UTF-8]{ctexbeamer}
\usetheme{Boadilla}

\usepackage{listing}
\usepackage[cache=false]{minted}

\title{First-class async in rCore}
\subtitle{设计方案}

\author{g01 - 王宇逸, 刘晓义}
\date{2020-3}

\begin{document}
\begin{frame}
  \titlepage
\end{frame}

\begin{frame}[fragile]
  \frametitle{Rust的Future架构}

  Rust只提供Future等一系列基础类型,最重要的是Future。
  {
  \scriptsize
  \begin{minted}{rust}
    pub enum Poll<T> {
      Ready(T),
      Pending,
    }

    pub trait Future {
      type Output;
      fn poll(self: Pin<&mut Self>, cx: &mut Context) -> Poll<Self::Output>;
    }

    pub struct Context<'a> { /* fields omitted */ }

    impl<'a> Context<'a> {
      pub fn from_waker(waker: &'a Waker) -> Context<'a>;
      pub fn waker(&self) -> &'a Waker;
    }
  \end{minted}
  }
\end{frame}

\begin{frame}[fragile]
  \frametitle{Rust的Future架构 Cont.}

  Rust 有一个叫做 Async 的东西

  \begin{minted}{rust}
    async fn foo() -> u64 {
      let result = bar().await;
      println!("Got result: {}", result);
      result
    }
  \end{minted}

  \vspace{1em}

  我们希望能够用 Async 和 Future 来完成内核中的并行工作,包括:
  \begin{itemize}
    \item 处理同步互斥
    \item 驱动用户线程
    \item 处理中断
  \end{itemize}
\end{frame}

\begin{frame}[fragile]
  \frametitle{But why?}

  \textbf{因为 Future 很便宜。}

  使用 \texttt{async fn} 得到的 Future 会由编译器静态计算出所需的空间,然后被放到堆上,这样就不需要对其他线程分配内核栈,也不需要担心内核栈爆炸了。

  \vspace{1em}

  \textbf{因为 Future 很方便。}

  \textbf{内建}了通知机制,同时有很多 Combinator 可以用。如果需要在内核内实现一个请求超时:

  {
  \scriptsize
  \begin{minted}{rust}
    async fn read(f: File, buf: &mut [u8], timeout: Duration)
      -> Result<usize, Error> {
        select! {
          len = f.read_into(buf) => Ok(len),
          () => timer::timeout(timeout) => Err(Error::Timeout),
        }
      }
  \end{minted}
  }
\end{frame}

\begin{frame}
  \frametitle{Previously...}

  有哪些有关的社区工作呢?
  \pause
  \begin{itemize}
    \item \textbf{futures:} 提供了很多 Combinators,虽然和 Scheduler 无关,但是在构建 Primitive Future 的时候,因为没办法直接用 async fn,这些组合子基本是一定要用的。
          \pause
    \item \textbf{Tokio \& async-std \& futures 的 executor:} 依赖 \texttt{std::sync::Mutex} 进行同步,还有 \texttt{std::thread::Thread} 创建进程池。其中后者在 Scheduler 内部连替代都写不出来。
          \pause
    \item \textbf{很多 no-std executor}: 只是在调用线程上同步执行,而且由于没有 Sync primitives,所以也没有异步消息。
          \pause
    \item \textbf{async-task:} 不是一个完整的 Executor,而是 Task 的抽象,处理了 Wake up 相关的逻辑。具体谁来 Poll,以及按照什么顺序 Poll,需要另外完成。
  \end{itemize}

\end{frame}
\begin{frame}
  \frametitle{Previously... Cont.}
  可以用 futures 的组合字和 async-task 的 Task 抽象,部分 Task 的组件 (Waker, etc.) 性能可能稍逊色于手动实现,除此以外还需要实现:
  \begin{itemize}
    \item 最基本的异步单元:Timeout, Mutex, Channel, ...
    \item Run queue 本身
  \end{itemize}
\end{frame}

\begin{frame}[fragile]
  \frametitle{基础设施}

  \begin{itemize}
    \item softirq: 中断队列
    \item timer: 计时器
    \item userspace manager: 将用户态对内核态的接口包装为 Future
    \item synchronization primitives: Mutex, Condvar, Channels...
    \item executor: 执行 Future
  \end{itemize}

\end{frame}

\begin{frame}[fragile]
  \frametitle{基础设施 Cont.}

  这些模块提供以下的 Primitives
  \vspace{1em}

  {
    \scriptsize
    \begin{minted}{rust}
    fn irq::wait_for(int: Interrupt) -> impl Future<Item = InterruptInfo>;
    fn timer::timeout(dur: Duration) -> impl Future<Item = InterruptInfo>;

    enum ut::Event { Syscall(Syscall), TimeUp, Signal(Signal) }
    impl ut::UserThread {
      fn run() -> impl Future<Item = Event>;
    }

    impl<T: Sync> sync::Mutex<T> {
      fn wait_until<F: Fn(&mut self, &mut T) -> bool>(pred: F)
        -> impl Future<Item = Guard<T>>;
    }

    impl exec::Executor {
      fn spawn(fut: Future<Item = ()>);
    }
    fn exec::yield() -> impl Future<Item = ()>;
  \end{minted}
  }
\end{frame}

\begin{frame}[fragile]
  \frametitle{Example}

  \scriptsize
  \begin{minted}{rust}
    async fn create_user_thread(us: UserThread, sig: Signals) {
      match us.run().await {
        Event::TimeUp => yield().await;
        Event::Syscall(Syscall::Exit(code)) => {
          us.notify_waiter(/* ... */);
          return;
        },
        Event::Syscall(Syscall::Sleep(spec)) => {
          let dur: Duration = spec.into();
          timer::timeout(dur).await;
        }
        Event::Signal(Signal::Kill) => {
          us.notify_waiter(/* ... */);
          return ();
        }
        // ...
      }
    }
  \end{minted}
\end{frame}

\begin{frame}[fragile]
  \frametitle{SleepFuture的实现}

  如何把Future与已经实现的timer结合起来?首先需要一个具体的SleepFuture结构。
  {
  \scriptsize
  \begin{minted}{rust}
    struct SleepFuture {
      pub shared_state: Arc<Mutex<SleepSharedState>>,
    }

    struct SleepSharedState {
      completed: bool,
      waker: Option<Waker>,
    }

    impl SleepFuture {
      pub fn new() -> Self {
        let shared_state = Arc::new(Mutex::new(SleepSharedState {
          completed: false,
          waker: None,
        }));
        SleepFuture {
          shared_state: shared_state,
        }
      }
    }
  \end{minted}
  }
\end{frame}

\begin{frame}[fragile]
  \frametitle{SleepFuture的实现}

  这是一个Future的通用实现,后面也可以考虑重用这部分代码来写其它的Future。
  {
  \scriptsize
  \begin{minted}{rust}
    impl Future for SleepFuture {
      type Output = ();
      fn poll(self: Pin<&mut Self>, cx: &mut FutureContext<'_>) -> Poll<Self::Output> {
        let mut shared_state = self.shared_state.lock();
        if shared_state.completed {
          Poll::Ready(())
        } else {
          shared_state.waker = Some(cx.waker().clone());
          Poll::Pending
        }
      }
    }    
  \end{minted}
  }
\end{frame}

\begin{frame}[fragile]
  \frametitle{SleepFuture的实现}

  {
    \scriptsize
    \begin{minted}{rust}
    pub(crate) fn tick(&self, cpu_id: usize, tid: Option<Tid>) -> bool {
      // ...
      Event::Wakeup(tid) => {
        let mut states = self.timer_states.lock();
        let shared_state = states.remove(&tid);
        if let Some(shared_state) = shared_state {
          let mut shared_state = shared_state.lock();
          self.set_status(tid, Status::Ready);
          if let Some(waker) = shared_state.waker.take() {
            waker.wake()
          }
        }
      }
      // ...
    }
  \end{minted}
  }

  这只是一个初步的实现。进一步可以考虑将ThreadPool改写,因为一个线程可能不止一个async任务,因此需要将调度的单位从Tid转为Waker。
\end{frame}

\begin{frame}
  \frametitle{Progress?}
  \begin{itemize}
    \item Timer 和 Queue 的数据结构以及相应的性能测试。
    \item 实现了 Timer 部分暴露的异步接口。
    \item 实现了一个 \texttt{no\_std} 下的单线程 Executor。
    \item API 设计
    \item 调查了之前社区进行的工作
  \end{itemize}

  \pause
  \vspace{2em}

  我们目前的分工是
  \begin{description}
    \item[王宇逸] 基本异步组件的实现,相关的数据结构
    \item[刘晓义] Executor 和调度器算法,相关的数据结构
  \end{description}

  未来在编写上层应用的时候再进行进一步的分工。
\end{frame}

\end{document}
